\documentclass{article}
\usepackage[utf8]{inputenc}
\usepackage{longtable}
\usepackage{a4wide}
\usepackage{amsmath}
\usepackage{amssymb}
\usepackage{pifont}
\usepackage{pdflscape}
\usepackage{graphicx}
\usepackage{xcolor}

\title{SOAP -- Spherical Overdensity and Aperture Processor}
\author{Bert Vandenbroucke, Joop Schaye, John Helly, Matthieu Schaller}
\date{August 2022}

\begin{document}

\maketitle

\section{Introduction}

SOAP computes four different types of properties, depending on whether all particles or only bound particles 
are included, and depending on how particles are included (by radius, in projection...). For all types, we use 
the halo membership and centre of potential as determined by VELOCIraptor (VR; Elahi et al., 2019). The latter 
is equivalent to the position of the most gravitationally bound particle in the halo. VR determines the 
binding energy by summing the gravitational potential energy and the kinetic energy. For gas particles, the 
thermal energy is also included in the calculation. These choices are not configurable. VR identifies 
(sub-)halos using a FOF algorithm in 6D phase space, and (sub-)halo membership is based on FOF membership. 
After the initial FOF search, VR runs a gravitational unbinding algorithm to remove particles that are part of 
the 6D FOF group but not actually gravitationally bound to the structure. This leaves us with two classes of 
(sub-)halo member particles: FOF particles, and FOF particles that are also bound. Note that VR does not know 
about neutrinos; neutrino particles can hence never be bound to a (sub-)halo.

\section{Property types}

\paragraph{Subhalo quantities (SH)} are computed for each (sub-)halo identified by VR, irrespective of whether 
this is a field halo or a satellite (or even satellite of satellite and so on). They include all particles 
that are members of the friends-of-friends (FOF) halo, or only those particles that are bound to the 
(sub-)halo (FOF minus unbound particles). For each (sub-)halo, there are thus only two sets of subhalo 
properties of this type. Subhalo properties are contained within the groups \verb+FOFSubhaloProperties+ and 
\verb+BoundSubhaloProperties+ in the output file, depending on whether they include all particles in the FOF 
or only the subset that is also gravitionally bound.

\paragraph{Exclusive sphere quantities (ES)} are similar to subhalo quantities, but they include only the 
particles that are bound to the (sub-)halo, and apply an additional radial cut (aperture). We use eight 
different aperture radii (10, 30, 50, 100, 300, 500, 1000, 3000 kpc), so that every (sub-)halo has eight of 
these. Exclusive sphere properties are contained within a group \verb+ExclusiveSphere/XXXkpc+, where 
\verb+XXX+ is the corresponding aperture radius.

\paragraph{Inclusive sphere quantities (IS)} use the same physical aperture radii as the exclusive sphere 
quantities, but include all particles within the radius, regardless of their membership status. They are
stored within a group \verb+InclusiveSphere/XXXkpc+.

\paragraph{Exclusive projected quantities (EP)} are similar to exclusive sphere quantities, except that their 
aperture filter is applied in projection, and this for independent projections along the x-, y- and z-axis. 
Along the projection axis, we do not apply any radial cut, so that the depth corresponds to all particles 
bound to the (sub-)halo. With four projected aperture radii (10, 30, 50, 100 kpc), we then have twelve sets of 
projected aperture quantities for each (sub-)halo. Projected aperture quantities are stored in a group named 
\verb+ProjectedAperture/XXXkpc/projP+, where \verb+XXX+ is the corresponding aperture radius, and \verb+P+ 
corresponds to a particular projection direction (\verb+x+, \verb+y+ or \verb+z+).

\paragraph{Spherical overdensity properties (SO)} are fundamentally different from the three other types in 
that their aperture radius is determined from the density profile and is different for different halos. They 
always include all particles within a sphere around the centre of potential, regardless of halo membership. 
The radius is either the radius at which the density reaches a certain target value (50 crit, 100 crit, 200 
crit, 500 crit, 1000 crit, 2500 crit, 200 mean, BN98) or a multiple of such a radius (5xR 500 crit). Details 
of the spherical overdensity calculation are given at the end of this document. Spherical overdensities are 
only computed for centrals, i.e. field halos. The inclusive sphere quantities are stored in a group 
\verb+SO/XXX+, where \verb+XXX+ can be either \verb+XXX_mean+ for density multiples of the mean density, 
\verb+XXX_crit+ for density multiples of the critical density, \verb+BN98+ for the overdensity definition of 
Bryan \& Norman (1998), and \verb+YxR_XXX_ZZZ+ for multiples of some other radius (e.g. \verb+5xR_2500_mean+). 
The latter can only be computed after the corresponding density multiple SO radius has been computed. This is 
achieved by ordering the calculations.

\paragraph{Some properties are directly copied from the original VR catalogue.} These are stored in a
separate group, \verb+VR+.

\paragraph{The table below summarises} the different halo types.

\begin{longtable}{llcp{6cm}}
 & Name & Inclusive? & Variations \\
\hline{}SH & \verb+XXSubhaloProperties+ & \ding{53} & \verb+FOF+, \verb+Bound+ \\
ES & \verb+ExclusiveSphere/XXXkpc+ & \ding{53} & 30, 50, 100, 300, 500, 1000, 3000 kpc \\
IS & \verb+InclusiveSphere/XXXkpc+ & \ding{51} & 30, 50, 100, 300, 500, 1000, 3000 kpc \\
EP & \verb+ProjectedAperture/XXXkpc/projP+ & \ding{53} & 10, 30, 50, 100 kpc \\
SO & \verb+SO/XXX+ & \ding{51} & 200 mean, (50, 100, 200, 500, 1000, 2500) crit, 5$\times{}$500 crit \\
- & \verb+VR/+ & - & Directly copied from VR output \\
\end{longtable}

\section{Property categories}

Halo properties only make sense if the (sub)halo contains sufficient particles. VR was run with a 
configuration that requires at least 20 particles within a satellite subhalo and 32 particles within a central 
subhalo. However, even for those particle numbers, a lot of the properties computed by SOAP will be zero (e.g. 
the gas mass within a 10 kpc aperture), or have values that are outliers compared to the full halo population 
because of undersampling. We can save a lot of disk space by filtering these out by applying appropriate cuts. 
Filtering means setting the value of the property to \verb+NaN+; HDF5 file compression then very effectively 
reduces the data storage required to store these properties, while the size of the arrays that the end user 
sees remains unchanged. Evidently, we can also save on computing time by not computing properties that are 
filtered out.

Since different properties can have very different requirements, filtering is done in categories, where each 
category corresponds to a set of quantities that are filtered using the same criterion. Inclusive, exclusive 
or projected quantities with different aperture radii (or overdensity criteria) can be used to create 
profiles. In order for these profiles to make sense, we have to apply a consistent cut across all the 
different aperture radii (or overdensity criteria) for the same subhalo property type. Or in other words: the 
quantities for an inclusive sphere with a 10 kpc aperture radius will use the same filter mask as the 
quantities of the inclusive sphere with a 3000 kpc aperture radius, even though the latter by construction has 
many more particles.

\paragraph{Basic quantities (basic)} are never filtered out, and hence go all the way down to the 20/32 
particle limit imposed by VR.

\paragraph{General derived quanties (general)} use a filter based on the total number of particles within the 
VR FOF group.

\paragraph{Derived gas quantities (gas)} use a filter based on the number of gas particles within the VR FOF 
group.

\paragraph{Derived DM quantities (dm)} use a filter based on the number of DM particles within the VR FOF 
group.

\paragraph{Derived stellar quantities (star)} use a filter based on the number of star particles within the VR 
FOF group.

\paragraph{Derived baryon quantities (baryon)} use a filter based on the number of gas and star particles 
within the VR FOF group.

\paragraph{}Note that there are no quantities that use a BH or neutrino particle number filter.

The different categories are summarised in the table below.

\begin{longtable}{ll}
Name & criterion \\
\hline{}basic & (all halos) \\
general & $N_{\rm{}gas}+N_{\rm{}dm}+N_{\rm{}star}+N_{\rm{}BH} \geq{} 100$ \\
gas & $N_{\rm{}gas} \geq{} 100$ \\
dm & $N_{\rm{}dm} \geq{} 100$ \\
star & $N_{\rm{}star} \geq{} 100$ \\
baryon & $N_{\rm{}gas}+N_{\rm{}star} \geq{} 100$ \\
\end{longtable}

\section{Overview table}

The table below lists all the properties that are computed by SOAP. This table is automatically generated by 
SOAP from the source code, so that all names, types, units, categories and descriptions match what is actually 
used and output by SOAP. For each quantity, the table indicates for which halo types the property is computed. 
Superscript numbers refer to more detailed explanations for some of the properties and match the numbers in 
the next section. The properties at the end of the table that are categorised as `VR' correspond to properties 
that are directly copied from the VR catalogue and are only listed for completeness. These do not belong to 
any type. Properties colored violet are also computed when SOAP is run in dark matter only (DMO) mode.

Note that all units are always physical, and that the SOAP output contains all the relevant meta-data to 
convert between physical and co-moving units, i.e. information about how the quantity depends on the 
scale-factor and what the conversion factor to and from CGS units is. All quantities are $h$-free.

\input{table}

\section{Non-trivial properties}

\input{footnotes}

\section{Spherical overdensity calculations}

The radius at which the density reaches a certain threshold value is found by linear interpolation of the 
cumulative mass profile obtained after sorting the particles by radius. The approach we use is different from 
that taken by VR, where both the mass and the radius are obtained from independent interpolations on the mass 
and density profiles (the latter uses the logarithm of the density in the interpolation). The VR approach is 
inconsistent, in the sense that the condition

\begin{equation}
    \frac{4\pi{}}{3} R_{\rm{}SO}^3 \rho{}_{\rm{}target} = M_{\rm{}SO},\label{eq:MSO_condition}
\end{equation}

is not guaranteed to be true, and will be especially violated for large radial bins (the bins are generated 
from the particle radii by sorting the particles, so we have no control over their width). We instead opt to 
guarantee this condition by only finding $R_{\rm{}SO}$ or $M_{\rm{}SO}$ by interpolation and using eq. 
(\ref{eq:MSO_condition}) to derive the other quantity.

\begin{figure}
    \centering
    \includegraphics[width=0.48\textwidth{}]{image7.png}
    \includegraphics[width=0.48\textwidth{}]{image4.png}
    \caption{Density profile (\emph{top row}) and cumulative mass profile (\emph{bottom row}) for an example 
    halo in a 400 Mpc FLAMINGO box. The orange lines show $\rho{}_{\rm{}target}$ and $R_{\rm{}SO}$ and 
    $M_{\rm{}SO}$ as determined by SOAP, while the green line is the cumulative mass profile at fixed 
    $\rho{}_{\rm{}target}$. The two left columns correspond to a run where $R_{\rm{}SO}$ is fixed by 
    interpolating on the density profile (so in the top row plot), while the second two columns determine 
    $R_{\rm{}SO}$ by interpolating on the cumulative mass in the bottom row plots. The right column for each 
    pair of columns shows a zoom of the left column.}
    \label{fig:MSO_vs_RSO}
\end{figure}

While the interpolation of the logarithmic density profile to find $R_{\rm{}SO}$ is more straightforward, we 
found that it can lead to significant deviations between the value of $M_{\rm{}SO}$ and the cumulative mass in 
neighbouring bins that can be more than one particle mass, as illustrated in Fig. \ref{fig:MSO_vs_RSO}. The 
reason for this is that the cumulative mass profile at fixed density increases very steeply with radius, so 
that a small difference in $R_{\rm{}SO}$ leads to a relatively large difference in $M_{\rm{}SO}$. Conversely, 
fixing $M_{\rm{}SO}$ will lead to an $R_{\rm{}SO}$ that only deviates a little bit from the $R_{\rm{}SO}$ 
found by interpolating the density profile. However, doing so requires us to find the intersection of the 
cumulative mass profile at fixed density (green line in Fig. \ref{fig:MSO_vs_RSO}) with the actual cumulative 
mass profile, which means solving the following equation:

\begin{equation}
    \frac{4\pi{}}{3} \rho{}_{\rm{}target} R_{\rm{}SO}^3 = M_{\rm{}low} + \left( \frac{M_{\rm{}high}-M_{\rm{}low}}{R_{\rm{}high} - R_{\rm{}low}} \right) \left(R_{\rm{}SO} - R_{\rm{}low}\right),
    \label{eq:RSO}
\end{equation}

where $R/M_{\rm{}low/high}$ are the bounds of the intersecting bin (which we find in the density profile). 
This third degree polynomial equation has no unique solution, although in practice only one of the three 
possible complex solutions is real. We find this solution by using a root finding algorithm within the 
intersecting bin (we use Brent's method for this).

For clarity, this is the full set of rules for determining the SO radius in SOAP:
\begin{enumerate}
    \item Sort particles according to radius and construct the cumulative mass profile.
    \item Discard any particles at zero radius, since we cannot compute a density for those. The mass of these 
    particles is used as an $r=0$ offset for the cumulative mass profile. Since the centre of potential is the 
    position of the most bound particle, there should always be at least one such particle.
    \item Construct the density profile by dividing the cumulative mass at every radius by the volume of the 
    sphere with that radius.
    \item Find intersection points between the density profile and the target density, i.e. the radii 
    $R_{1,2}$ and masses $M_{1,2}$ where the density profile goes from above to below the threshold:
    \begin{enumerate}
        \item If there are none, analytically compute $R_{\rm{}SO}=\sqrt{3M_1/(4\pi{}R_1\rho_{\rm{}target})}$, 
        where $R_1$ and $M_1$ are the first non zero radius and the corresponding cumulative mass. This is a 
        special case of Eq. (\ref{eq:RSO}). Unless there are multiple particles at the exact centre of potential 
        position, this radius estimate will then be based on just two particles.
        \item In all other cases, we use $R_{1,2}$ and $M_{1,2}$ as input for Eq. (\ref{eq:RSO}) and solve for 
        $R_{\rm{}SO}$. The only exception is the special case where $R_1 = R_2$. If that happens, we simply 
        move further down the line until we find a suitable interval.

    \end{enumerate}
    \item From $R_{\rm{}SO}$, we determine $M_{\rm{}SO}$ using Eq. (\ref{eq:MSO_condition}).
\end{enumerate}

\paragraph{Neutrinos -- if present in the model -- are included} in the inclusive sphere calculation (and only 
here, since neutrino particles cannot be bound to a halo) by adding both their weighted masses (which can be 
negative), as well as the contribution from the background neutrino density. The latter is achieved by 
explicitly adding the cumulative mass profile at constant neutrino density to the total cumulative mass 
profile before computing the density profile. This is the only place where neutrinos explicitly enter the 
algorithm, except for the neutrino masses computed for the SOs. Neutrinos are not included in the calculation 
of the centre of mass and centre of mass velocity.

\end{document}
